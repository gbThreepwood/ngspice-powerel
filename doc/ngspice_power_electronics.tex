\documentclass[10pt,a4paper]{article}
\usepackage[T1]{fontenc}
\title{Ngspice for kraftelektronikk}
\author{Eirik Haustveit}
\begin{document}
	\maketitle
	
	
	Dette kapittelet tar for seg bruken av Ngspice til simulering av tilnærma ideelle omformarar. Formålet er mellom anna å gera lesaren i stand til å simulera idealiserte omformarar for å skaffa seg betre innsikt i verkemåten til topologiane, og til å verifisera teoretiske berekningar.
	
	
	\section{Lineære krestelement med sinusforma straum og spenning}
	
	
	\subsection{Teikning av grafar med gnuplot}
	
	\subsection{Teikning av grafar i ekstern programvare}
	
	
	\subsection{Måling av ulike verdiar (Snitt, RMS, etc.)}
	
	
	\section{Ikkje-sinusforma straum og spenning}
	
	
	\subsection{Fourieranalyse}

	
	\section{Diodelikerettar}
	
	
	\subsection{Modellering av ein ideell diode}
	
	
	
	\section{Tyristoromformar}
	
	
	\subsection{Modellering av ein ideell tyristor}
	
	\subsection{Generering av synkronisert pulssignal for tyristor}
	
	Sidan ein normalt starter simuleringa ved tida \(t = 0\) vil ei spenningskjelde med pulsspenning starta opp i fase sinusspenninga som forsyner tyristoromformaren. Dette er derimot ikkje eit realistisk scenario. I praksis vil det vera heilt tilfeldig kvar du er på sinuskurva når regulatoren for tyristoromformaren starter opp, og du er derfor avhengig av å måla tilført spenning for å finna ut når du skal trigga ein gitt tyristor. Denne synkroniseringa vil ofte gjennomførast ved hjelp av ein nullgjennomgangsdetektor, eller ved hjelp av ei faselåst sløyfe (PLL).
	
	\subsection{Nullgjennomgangsdetektor}
	
	Dersom ein kun er ute etter å simulera ein idealisert krins, så kan det vera tilstrekkelig å laga ein firkantpuls som er høg så lenge sinuskurva er i sin positive halvperiode. Ei flanketrigga vippe kan 
	
	\subsubsection{Faselåst sløyfe (PLL)}
	
	Ein PLL består av (minst) tre primærkomponentar:
	
	\begin{itemize}
		\item Spenningsstyrt oscillator (VCO)
		\item Fasedetektor
		\item Sløyfefilter
	\end{itemize}
	
	
	
	\section{DC/DC omformar}
	
	
	\subsection{Modellering av ein ideell styrt brytar}
	
	\subsection{Generering av PWM med variabel arbeidssyklus}
	
	
	\section{DC/AC omformar}
	
	
	\subsection{Generering av sinusforma PWM}
	
	
	
	

	

	
	
\end{document}